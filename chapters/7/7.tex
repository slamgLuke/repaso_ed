\documentclass{report}

\input{../../template/preamble.tex}
\input{../../template/macros.tex}
\input{../../template/letterfonts.tex}

\title{\Huge{Coeficientes indeterminados}}
\author{\huge{Lucas Carranza}}
\date{}

\begin{document}

\maketitle
\newpage

\chapter{}
\section{Coeficientes indeterminados - Método de superposición}
\dfn{}{\textbf{DEFINICI\'ON:} Método para resolver ecuaciones diferenciales lineales no-homogéneas de segundo orden con coeficientes constantes. Se identifica la forma de la solución particular correspondiente a la parte no-homogénea, y se reemplaza por la función desconocida $(y)$ en la parte homogénea para obtener las constantes.}

\subsection{Pasos}

\begin{enumerate}
\item Identificar la forma de la solución particular correspondiente a la parte no-homogénea $f(x)$.

\item Reemplazar la solución particular en la parte homogénea para obtener las constantes $y = f(x)$.

\item Sumar la solución particular (con constantes halladas) con la solución homogénea [FIN].
\end{enumerate}

\subsection{Tabla de soluciones particulares}

% imagen de la tabla
\begin{figure}[h]
    \centering
    \includegraphics[width=0.8\textwidth]{tabla.png}
\end{figure}

\end{document}
