\documentclass{report}

\input{../../template/preamble.tex}
\input{../../template/macros.tex}
\input{../../template/letterfonts.tex}

\title{\Huge{Coeficientes indeterminados - Método del operador anulador}}
\author{\huge{Lucas Carranza}}
\date{}

\begin{document}

\maketitle
\newpage

\chapter{}
\section{Coeficientes indeterminados - Método del operador anulador}
\dfn{}{\textbf{Operador anulador:} Es un operador diferencial que al ser aplicado a una función, la vuelve 0.}




\dfn{}{\textbf{Método del operador anulador:} Es un método para resolver ecuaciones diferenciales no homogéneas. Consiste en encontrar un operador anulador que al ser aplicado a la ecuación diferencial, la vuelve homogénea (se anula la parte no-homogénea). Luego, se resuelve la ecuación homogénea y se obtiene la solución general.}

\subsection{Pasos}

\begin{enumerate}

\end{enumerate}

\nt{\textbf{No hace falta} que la ecuación esté en la forma estándar.}

\subsection{}


\subsection{Ejemplo polinomial}
\ex{Resuelva la ecuación diferencia: $y'' $}{

}

\subsection{Ejemplo exponencial}

\subsection{Ejemplo trigonométrico}

\subsection{Ejemplo combinado}

\end{document}
