\chapter{}
\section{Coeficientes indeterminados - Método del operador anulador}
\dfn{}{\textbf{Operador anulador:} Es un operador diferencial que al ser aplicado a una función, la vuelve 0.}

\subsection{Tabla de operadores anuladores}

\begin{center}
\begin{tabular}{|c|c|c|}
\hline
\textbf{f(x)} & \textbf{Operador anulador} \\
\hline
$x^n$ & $D^{n+1}$\\
\hline
$\sin(ax)$ & $(D^2+a^2)$ \\
\hline
$\cos(ax)$ & $(D^2+a^2)$ \\
\hline
$\cos(ax)e^{bx}$ & $(D^2-2abD+(a^2+b^2))$ \\
\hline
$x^n\sin(ax)e^{bx}$ & $( D^2-2abD+(a^2+b^2) )^{n+1}$ \\
\hline
$e^{ax}$ & $(D-a)$ \\
\hline
$x^n e^{ax}$ & $(D-a)^{n+1}$ \\
\hline
\end{tabular}
\end{center}

\subsection{Método}

\dfn{}{\textbf{Método del operador anulador:} Es un método para resolver ecuaciones diferenciales no homogéneas. Consiste en encontrar un operador anulador que al ser aplicado a la ecuación diferencial, la vuelve homogénea (se anula la parte no-homogénea). Luego, se resuelve la ecuación homogénea y se obtiene la solución general.}

\nt{\textbf{No hace falta} que la ecuación esté en la forma estándar.}

\subsection{Pasos}

\begin{enumerate}
    \item Identificar el operador anulador correspondiente a la parte no-homogénea $f(x)$.

    \item Aplicar el operador anulador de $f(x)$ a ambos lados de la ecuación diferencial.

    \item Obtener las raíces del polinomio característico (donde $D$ es la incógnita).

    \item Obtener la solución particular de la ecuación no-homogénea.

    \item Obtener la solución general de la ecuación homogénea.


    \item Sumar la solución particular con la solución general, omitiendo las soluciones duplicadas de la solución particular que ya estén en la homogénea [FIN].
\end{enumerate}


\subsection{Ejemplo polinomial}
\ex{Resuelva la ecuación diferencial: $y'' $}{

}

\subsection{Ejemplo exponencial}

\subsection{Ejemplo trigonométrico}

\subsection{Ejemplo combinado}

