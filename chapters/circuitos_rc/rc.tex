\documentclass{report}
\input{~/tex/preamble.tex}
\input{~/tex/macros.tex}
\input{~/tex/letterfonts.tex}
\usepackage{multicol}

\author{Lucas Carranza}
\title{Circuitos RLC 2}
\date{}

\begin{document}

\maketitle
\newpage


\chapter{}

\begin{multicols}{2}
[
\section{Datos nuevos}
]
\dfn{$\omega_0$}{
    Frecuencia de resonancia. ($rad/s$)
    \[\omega_0 = \frac{1}{\sqrt{LC}}\]
    \[\omega_0^2 = \frac{1}{LC}\]
}
\dfn{$f$}{
    Frecuencia de oscilación. ($Hz$)
    \[f = \frac{\omega_0}{2\pi}\]
    \[f = \frac{1}{2\pi\sqrt{LC}}\]
}
\end{multicols}

\nt{
    \begin{itemize}
        \item $L$ es la inductancia/bobina en \textit{Henrys}. 
        \item $C$ es el capacitor en \textit{Farads}.
    \end{itemize}
}

\section{Ecuación diferencial - CIRCUITO RC}
\thm{La ecuación diferencial de un sistema RC es}{
    \[LQ(t)'' + \frac{1}{C}Q(t) = 0\]

    Dividimos entre $L$ para dejarlo en forma estándar:
    \[Q(t)'' + \frac{1}{LC}Q(t) = 0\]

    Como sabemos que $\omega_0^2 = \frac{1}{LC}$, podemos escribir:
    \[Q(t)'' + \omega_0^2Q(t) = 0\]
}

\subsection{Solución general}
\[r^2 + \omega_0^2 = 0\]
\[r^2 = -\omega_0^2\]
\[r = 0 \pm i\omega_0\]

\nt{
    La solución general de la ecuación diferencial es:
    \[Q(t) = c_1\cos(\omega_0t) + c_2\sin(\omega_0t)\]
}

\section{Ecuación diferencial - CIRCUITO RLC}
\thm{}{
    Al añadir una resistencia $R$, se tiene una ecuación de amortiguamiento.
    La energía se disipa lentamente en forma de calor.
    \[LQ(t)'' + RQ(t)' + \frac{1}{C}Q(t) = 0\]

    Dividimos entre $L$ para dejarlo en forma estándar:
    \[Q(t)'' + \frac{R}{L}Q(t)' + \omega_0^2(t) = 0\]
}

\subsection{Amortiguamiento crítico}
\[r^2 + \frac{R}{L}r + \frac{1}{LC} = 0\]
\[r = \cfrac{-\cfrac{R}{L} \pm \sqrt{\Delta}}{2}\]

\nt{
    \begin{itemize}
        \item Para que sea sub-amortiguado, $\Delta < 0$. 
        \item Para que sea sobre-amortiguado, $\Delta > 0$.
        \item Para que sea críticamente amortiguado, $\Delta = 0$.
    \end{itemize}
}

\[\Delta = \left(\frac{R}{L}\right)^2 - 4\frac{1}{LC} < 0\]
\[\frac{R^2}{L^2} < \frac{4}{LC}\]
\[R < 2\sqrt{\frac{L}{C}}\]


\end{document}
